% chapter included in forwardcom.tex
\documentclass[forwardcom.tex]{subfiles}
\begin{document}
\chapter{Программируемые специфичные для приложения команды}
Вместо реализации огромного количества специальных команд для конкретных приложений, мы можем предоставить средства порождения определяемых пользователем команд, которые могут быть закодированы на языке описания аппаратуры, например, VHDL или Verilog. 

Микропроцессор может иметь необязательную ПЛИС или подобную программируемую аппаратуру. Эта структура может использоваться для создания специфичных для приложений команд или функций, например, кодирования, шифрования, сжатия данных, обработки сигналов, обработки текстов, и т.п.

Если у процессора имеется несколько ядер, то каждое ядро может иметь свою собственную ПЛИС. Код определения аппаратуры для каждого ядра сохраняется в своём собственном кэше. Операционной системе следует предотвращать, насколько возможно долго, использование одного и того же ядра разными задачами, требующими разного аппаратного кода. Могут быть черты, позволяющие приложению монополизировать ПЛИС или её часть.

Если невозможно избежать использование одной и той же ПЛИС на одном и том же ядре несколькими приложениями, то код, также как содержимое всех ячеек памяти ПЛИС, должен сохраняться при переключении задач. Сохранение может быть реализовано как ленивое, т.е. содержимое переключается только тогда, когда второй задаче нужна структура ПЛИС, содержащая код первой задачи.

Должны быть команды для обращения к определённым пользователем функциям, включая средства ввода и вывода, и для адаптации задержки определённых пользователем функций.
\end{document}
