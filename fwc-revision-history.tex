% chapter included in forwardcom.tex
\documentclass[forwardcom.tex]{subfiles}
\begin{document}
\chapter{История исправлений}
\subsubsection{Версия 1.02, 2016-06-25.}
\begin{itemize}
\item Название изменено на ForwardCom.	
\item Перемещено на GitHub.
\item Добавлены разные вопросы, связанные с безопасностью.
\item Поддержка двух стеков.
\item Модифицированы некоторые форматы команд, включая дополнительные форматы для команд jump и call.
\item Добавлены команды системного вызова, системного возврата, и ловушки.
\item Новый адресный режим для массивов с проверкой границ.
\item Модифицировано или добавлено несколько команд.
\item Управление памятью и стандарты ABI описаны более детально.
\item Список команд в файле instruction\_list.csv, с запятой в качестве разделителя.
\item Формат объектного файла определён в файле elf\_forwardcom.h
\end{itemize}

\subsubsection{Версия 1.01, 2016-05-10.}
\begin{itemize}
\item Набору команд дано имя CRISC1.
\item Длина векторного регистра сохраняется в самом регистре. Как следствие, модифицируется базовая структура кода. Также, как следствие, упрощаются соглашения о вызове функций.
\item Определены все команды уровня пользователя.
\item Весь текст был переписан и обновлён.
\end{itemize}

\subsubsection{Версия 1.00, 2016-03-22.}
Данный документ --- результат длительного обсуждения на сайте \href{http://www.agner.org/optimize/blog/read.php?i=421}{Agner Fog's blog}, начавшегося  2015-12-27, а также в списке рассылки RISC-V и на форуме Opencores

Дополнительное воодушевление было найдено в разных источниках, перечисленных на с.~\pageref{referencesToIntroduction}. 

Версия 1.00 данного руководства была опубликована на сайте \href{http://www.agner.org/optimize}{www.agner.org/optimize}.
\end{document}