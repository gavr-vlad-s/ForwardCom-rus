% chapter included in forwardcom.tex
\documentclass[forwardcom.tex]{subfiles}
\begin{document}
%\raggedright
%\RaggedRight

\chapter{Введение}
ForwardCom означает Forward Compatible Computer system. 

Данный документ предлагает новую открытую архитектуру набора команд, спроектированную для оптимальной производительности, гибкости и масштабируемости.
Проект ForwardCom включает в себя как новую архитектуру набора команд, так и соответствующую экосистему программных стандартов --- прикладной двоичный 
интерфейс (ABI) , управление памятью, средства разработки, форматы библиотек и системные функции. Этот проект иллюстрирует улучшения, которых можно 
добиться посредством полного вертикального перепроектирования аппаратного и программного обеспечения, основываясь на открытом совместном процессе.

Настоящее руководство и весь связанный с ним код находятся по адресу \href{https://github.com/ForwardCom/}{https://github.com/ForwardCom}.
 
\section{Основные моменты}
\begin{itemize}
\item Набор команд ForwardCom является компромиссом между принципами RISC и CISC, объединяя быстрый и оптимизированный дизайн декодирования и конвейера
RISC--систем с компактностью и б\'{о}льшим числом работы, в пересчёте на одну команду, CISC--систем.

\item Дизайн ForwardCom --- масштабируется, для поддержки как малых встроенных систем, так и огромных суперкомпьютеров и векторных процессоров, без 
потери двоичной совместимости.

\item Для обработки больших наборов данных предоставляются векторные регистры переменной длины.

\item Циклы по массивам реализованы новым гибким способом, автоматически использующим максимальную длину вектора, поддерживаемую микропроцессором, во
всех итерациях цикла, кроме последней. Последняя итерация автоматически использует такую длину вектора, в которую целиком помещается оставшееся число
элементов. Для обработки с оставшихся данных и специальных случаев не нужно никакого дополнительного кода. Нет нужды компилировать код отдельно для 
разных микропроцессоров с разными длинами векторов.

\item Когда становится доступным новый микропроцессор, с большей длины векторными регистрами, не нужно перекомпиляции или обновления программного обеспечения. Программное обеспечение гарантированно будет совместимым снизу вверх, и получит преимущество от более длинных векторов новых моделей микропроцессора.

\item Сильные функции безопасности --- фундаментальная часть аппаратного и программного дизайна.

\item Управление памятью проще и эффективнее, чем в традиционных системах. Используются различные приёмы для устранения фрагментации памяти. Нет разбиения  памяти на страницы и буфера ассоциативной трансляции (translation lookaside buffer, TLB). Вместо этого есть отображение памяти, с ограниченным количеством секций переменного размера.

\item Нет динамически компонуемых библиотек (dynamic link libraries, DLLs) или разделяемых объектов. Вместо этого есть лишь один тип библиотек функций,
который может использоваться и для статической, и для динамической компоновки. Загружается и компонуется лишь та часть библиотеки, которая действительно используется. Библиотечный код почти во всех случаях хранится непрерывно с кодом основной программы. В момент загрузки на основании аппаратной конфигурации, операционной системы, или окружения пользовательского интерфейса, можно автоматически выбирать разные версии функции или библиотеки.

\item Предоставляется механизм для вычисления требуемого размера стека, что может в большинстве случаев предотвратить переполнение стека, не создавая при этом стек большего размера, чем необходимо.

\item Предоставляется механизм оптимального распределения регистров по программным модулям и библиотекам функций. Это делает возможным сохранение большинства переменных в регистрах без сброса этих переменных в память. Векторные регистры могут быть сохранены эффективным способом, сохраняющим только ту часть регистра, которая действительно используется.
\end{itemize}

\section{Основы}
Архитектура набора команд --- это стандартизированный набор машинных команд, который может выполнять компьютер. Имеется много используемых архитектур наборов команд.

Некоторые широко используемые наборы команд плохо спроектированы с самого начала. Эти системы много раз дополнялись расширениями и заплатками. Один из 
наихудших случаев --- широко используемый набор команд x86 и его многочисленные расширения. Набор команд x86 --- результат длительной истории недальновидных расширений и заплаток. Результат этой разработки --- очень сложная архитектура, с тысячами разных кодов команд, очень сложная и дорогая для
декодирования микропроцессором. Мы должны извлечь уроки из прошлых ошибок, чтобы сделать более хороший выбор при проектировании новой архитектуры набора
команд и поддерживающего её программного обеспечения.

Дизайн должен быть основан на открытом процессе. Кырсте Асанович (Krste Asanović) и Дэвид Паттерсон (David Patterson) представили убедительные аргументы в пользу того, что следует предпочесть открытый набор команд. Открытость может быть решающей для успеха технического дизайна. Например, первоначальный IBM PC в начале 1980--х имел преимущество по сравнению с конкурирующими компьютерами, ибо открытая архитектура позволяла другим производителям аппаратного и программного обеспечения выпускать совместимое оборудование. IBM утратила своё доминирующее положение на рынке, когда она в 1987г. перешла к проприетарной Micro Channel Architecture. Также хорошо известен и не нуждается в дальнейшем обсуждении успех программного обеспечения с открытым исходным кодом. Единственно, что отсутствует в полной компьютерной экосистеме, основанной на открытых стандартах --- открытая микропроцессорная архитектура, откроющая рынок также для меньших производителей микропроцессоров и для нишевых продуктов.

Настоящее руководство основано на обсуждении на различных интернет--форумах. Спецификации являются предварительными. Разработка нового стандарта извлекла бы пользу из длинной экспериментальной фазы, и было бы неразумно зафиксировать стандарт на данной, начальной, стадии.

\section{Цели дизайна}
Ранее опубликованные открытые наборы команд подходят для малых, дешёвых микропроцессоров, предназначенных для встроенных систем, однокристалльных систем,
реализаций на основе ПЛИС для научных экспериментов, и т.п. Предлагаемая архитектура ForwardCom развивает эту идею, и нацелена на дизайн, который может
превзойти существующие высокопроизводительные процессоры.

Набор команд ForwardCom основан на следующих приоритетах:
\begin{itemize}
\item Набор команд должен обладать простым и последовательным модульным дизайном.

\item Набор команд должен представлять подходящий компромисс между принципами RISC, позволяющими быстрое декодирование, и принципами CISC, делающими
возможным выполнение большей работы в расчёте на одну команду и более эффективное использование кэша кода.

\item Дизайн должен быть расширяемым, чтобы новые команды и расширения можно было добавлять последовательным и предсказуемым образом. 

\item Дизайн должен быть масштабируемым, чтобы он подходил и для малых компьютеров с памятью на кристалле, и для огромных суперкомпьютеров с очень большими векторами.

\item Дизайн должен быть конкурентоспособен по сравнению с имеющимися коммерческими, и сосредоточен на высокопроизводительных приложениях завтрашнего дня, а не низкопроизводительных приложениях дня вчерашнего.

\item Поддержка векторов и других черт, которые, как давно доказано, существенны для высокой производительности, должна быть фундаментальной частью дизайна, а не неуклюжим отростком.

\item Фундаментальной частью дизайна должна быть безопасность, а не специально добавляемые заплатки.

\item Набор команд должен быть спроектирован в открытом процессе, с участием международного сообщества программистов и \glqq железячников\grqq, подобно работе по стандартизации в других технических областях.

\item Весь вертикальный дизайн должен быть непроприетарным, и должен позволять всем создавать создавать совместимое программное обеспечение, аппаратное обеспечение, и оборудование, для тестирования, отладки, и эмуляции.

\item Решения о командах и расширениях должны определяться не краткосрочными маркетинговыми соображениями монополистической микропроцессорной промышленности, а долгосрочными нуждами всего сообщества программистов, \glqq железячников\grqq, и организаций. 

\item Дизайн должен позволять построение совместимого снизу вверх программного обеспечения, которое будет работать без перекомпиляции на будущих процессорах с большего размера векторными регистрами.

\item Дизайн должен позволять специфичные для приложения расширения.

\item Базовые аспекты экосистемы --- стандарт для ABI, ассемблер, компиляторы, библиотеки функций, системные функции, каркас пользовательского интерфейса, и т.п. --- также, ради максимальной совместимости, должны быть стандартизированы.
\end{itemize}

Новому набору команд нелегко будет добиться успеха на коммерческом рынке, даже если этот набор лучше, нежели устаревшие системы, поскольку рынок предпочитает обратную совместимость с существующим программным и аппаратным обеспечением. Вряд ли набор команд ForwardCom станет успешным коммерческим продуктом, но обсуждение идеальных набора команд и программной экосистемы всё же могло бы быть полезным. Проект ForwardCom уже породил много важных идей, так что его стоит разрабатывать дальше, даже если мы не узнаем, когда он закончится. Настоящая работа может быть полезной, если по другим причинам должна возникнуть необходимость во введении нового набора команд. Работа будет особенно полезной для больших векторных процессоров; для приложений, в которых важна безопасность; для операционных систем реального времени; а равно и для проектов, в которых были бы препятствием патентные и лицензионные ограничения иных архитектур.

Предложения данного документа могут также быть полезны как источник вдохновения и для научных экспериментов. Многие идеи не зависят от деталей дизайна, и могут быть реализованы в существующих системах.

\section{Сравнение с другими открытыми наборами команд}
Было предложено несколько других открытых наборов команд, наиболее заметные --- RISC-V и OpenRISC. Оба имеют чистый RISC--дизайн, с, преимущественно, фиксированным 32--разрядными командными словами. Эти наборы команд подходят для малых систем, где экономится место на кремнии, но они не спроектированы для высокопроизводительных суперскалярных процессоров, и не сосредоточены на деталях, критичных для достижения максимальной производительности в б\'{о}льших системах. Настоящее предложение рассматривается как следующий шаг к созданию открытого набора команд, который действительно эффективнее наилучших сегодняшних коммерческих наборов команд.

Типичный RISC--дизайн с размером команды, ограниченным 32 разрядами, оставляет ограниченное пространство для непосредственно заданных констант и адресов находящихся в памяти операндов. Программе среднего размера потребуются 32--разрядные относительные адреса находящихся в статической памяти операндов, во избежание переполнения во время процесса переразмещения адресов компоновщиком. Тридцатидвухразрядный относительный адрес требует нескольких команд в чистых RISC--дизайнах. Например, для прибавления находящегося в памяти операнда к регистру в чистом RISC--дизайне с только 32--разрядными командными словами вам потребуется пять команд: (1) загрузить младшую часть 32--разрядного адресного смещения; (2) прибавить старшую часть 32--разрядного адресного смещения; (3) прибавить точку отсчёта или указатель команд к этому значению; (4) прочесть операнд, находящийся по вычисленному адресу памяти; (5) выполнить желаемое сложение. Дизайн ForwardCom выполняет все указанные действия одной командой с удвоенным размером слова. Преимущество в скорости очевидно. Вычисление адреса, загрузка, и выполнение, чтобы достичь плавного прохода одной команды за такт в каждой ветви конвейера, осуществляются на своих стадиях конвейера, 

Другое важное отличие состоят в том, что предыдущие RISC--дизайны имели ограниченную поддержку векторных операций. Дизайн ForwardCom вводит новую систему векторных регистров переменной длины, более эффективную и гибкую, чем наилучшие имеющиеся коммерческие дизайны. Эффективные векторные операции существенны для получения максимальной производительности, и были важным приоритетом в предлагаемом здесь дизайне ForwardCom.

\section{Литература и ссылки} \label{referencesToIntroduction}

\begin{itemize}
\item
Krste Asanovi\'{c} and David Patterson: ``The Case for Open Instruction Sets. Open ISA Would Enable Free 
Competition in Processor Design''. Microprocessor Report, August 18, 2014. \\
\href{http://www.linleygroup.com/mpr/article.php?id=11267}{www.linleygroup.com/mpr/article.php?id=11267}

\item
RISC-V: The Free and Open RISC Instruction Set Architecture
\href{http://riscv.org}{riscv.org}

\item
OpenRISC: 
\href{http://openrisc.io}{openrisc.io}

\item
Open Cores: 
\href{http://opencores.org}{opencores.org}

\item
Agner Fog: Proposal for an ideal extensible instruction set, 2015. A blog discussion thread that initiated the ForwardCom project. \\
\href{http://www.agner.org/optimize/blog/read.php?i=421}{www.agner.org/optimize/blog/read.php?i=421}

\item
Agner Fog: Stop the instruction set war, 2009. Blog post about the problems with the x86 instruction set. \\
\href{http://www.agner.org/optimize/blog/read.php?i=25}{www.agner.org/optimize/blog/read.php?i=25}

\item
Darek Mihocka: Standard Need To Be Forward Looking, 2007. Blog post criticizing the x86 instruction set standard. \\
\href{http://www.emulators.com/docs/nx02_standards.htm}{www.emulators.com/docs/nx02\_standards.htm}.
See also the following pages.
\end{itemize}

\end{document}