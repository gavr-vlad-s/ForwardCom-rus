% chapter included in forwardcom.tex
\documentclass[forwardcom.tex]{subfiles}
\begin{document}
%\RaggedRight

\chapter{Основы архитектуры}
В данной главе приводится обзор наиболее важных черт архитектуры набора команд ForwardCom. Детали приводятся в последующих главах.

\section{Полностью ортогональный набор команд}
Набор команд ForwardCom --- полностью ортогонален во всех отношениях. Одна и та же команда может использовать целочисленные операнды всех размеров и вещественные операнды всех точностей. Она может использовать регистровые операнды, операнды из памяти, или непосредственно заданные операнды. Она может также использовать много разных режимов адресации. Команды могут кодироваться в коротких формах с двумя операндами, в которых один и тот же регистр и как операнд--приёмник, и как операнд источник; или в более длинных формах с тремя операндами. Команда может работать со скалярами и векторами любого размера. Она может иметь предикаты или маски для условного выполнения для векторов на уровне элементов, и может иметь в качестве входных аргументов флаги --- для определения режима округления, управления исключениями, и других деталей, там, где это нужно. Константные данные всех типов могут включаться в команды, и, для уменьшения размера команды, сжиматься различными способами.

\subsubsection{Обоснование}
Ортогональность реализуется посредством стандартизированного модульного дизайна, что упрощает реализацию аппаратуры, а также делает компиляцию более простой и гибкой, и облегчает компилятору преобразование линейного кода в векторный.

Поддержка непосредственно заданных констант всех типов является улучшением, по сравнению с имеющимися системами. Большинство имеющихся систем хранят вещественные константы в сегменте данных, и обращаются к ним через 32--разрядные адреса в коде команды. Это является пустой тратой места в кэше данных и вызывает много промахов кэша, так как данные разбросаны по разным секциям. Замена 32--разрядного адреса 32--разрядной непосредственно заданной константой делает код более эффективным и не увеличивает размер кода. Расширения, позволяющие 64--разрядные непосредственно заданные константы, возможны ценой наличия команд тройной длины. Однако эта возможность в базовом дизайне ForwardCom не требуется, поскольку, по объяснённым ниже причинам, приоритет состоял в минимизации количества разных размеров команд.

\section{Размер команды}
Набор команд ForwardCom для кода использует 32--разрядное слово. Команда может состоять из одного или двух 32--разрядных слов, с возможным расширением на три или более слова. Плотность кода можно увеличить, используя малые команды половинного размера, а равная 32 разрядам единица размера сохраняется соединением малых команд по две. Невозможно перейти во вторую малую команду из такой пары малых команд. В будущем можно добавить расширения с командами размером в три или более слова.

\subsubsection{Обоснование}
Архитектура CISC с многими разными размерами команд --- неэффективна для суперскалярных процессоров, на которых мы хотим выполнять несколько команд за такт. Декодирующий препроцессор --- часто узкое место. Вы должны определить длину первой команды, прежде чем узнаете начало новой команды. \glqq Декодирование длин команд\grqq\ по своей сути --- последовательный процесс, что делает сложным декодирование нескольких команд за такт. У некоторых микропроцессоров, чтобы обойти это узкое место, есть дополнительный \glqq кэш микроопераций\grqq, находящийся после декодера.

Желательно иметь как можно меньше разных длин команд, и чтобы облегчить определение длины каждой команды. Мы хотим малый размер команды для наиболее употребимых простых команд, но нам также нужен больший размер команд, чтобы вместить вещи вроде большего набора регистров, команд с многими операндами, векторных операций с продвинутыми возможностями, 32--разрядных адресных смещений, и больших непосредственно заданных констант. Данное предложение --- компромисс между компактностью кода, лёгким декодированием, и пространством для продвинутых возможностей.

\section{Набор регистров}
Имеется 32 регистра общего назначения (r0--r31) по 64 разряда в каждом, и 32 векторных регистра (v0--v31) переменной длины. Максимальная длина вектора --- разная для разных реализаций аппаратуры. Регистры общего назначения можно использовать для целых чисел разрядности до 64 включительно и для указателей. Векторные регистры можно использовать для скаляров, либо для векторов, состоящих из целых и вещественных чисел.

Определяются следующие специальные регистры, видимые на уровне прикладной программы (все --- 64--разрядные):
\begin{itemize}
\item указатель команд (Instruction pointer, IP);
\item указатель секции данных (Data section pointer, DATAP);
\item указатель блока окружения потока (Thread environment block pointer THREADP);
\item указатель стека (Stack pointer, SP);
\item численный управляющий регистр (Numeric control register, NUMCONTR).
\end{itemize}

Указатель стека идентичен r31. К другим специальным регистрам как к обычным регистрам обращаться нельзя.

Специального регистра флагов нет. Регистры r1--r7 и v1--v7 могут использоваться для масок, предикатов, и флагов для вещественных чисел, для управления такими атрибутами, как режим округления и управление исключениями.

Неиспользуемая часть регистра всегда устанавливается равной нулю. Это означает, что целочисленные операции с размером операнда, меньшим 64 разрядов, и векторные операции с размером вектора, меньшим максимального, всегда обнуляют неиспользуемые разряды регистра--приёмника.

\subsubsection{Обоснование}
Количество регистров --- компромисс между плотностью кода и гибкостью. Цена сброса регистров в память обычно важна лишь в критичном наиболее глубоко вложенном цикле, которому вряд ли нужно более 32 регистров.

Мы можем избежать ложных зависимостей от предыдущего содержимого регистра, обнулив все неиспользуемые разряды регистра, вместо того, чтобы оставлять их неизменными. Аппаратура может сэкономить энергию, отключив неиспользуемые части исполнительных модулей и шин данных.

Специальный регистр флагов нежелателен для кода, планирующего вычисления как посредник между последними, и для векторного кода.

Причина обработки вещественных скаляров в векторных регистрах, а не в отдельных регистрах, заключается в облегчении для компилятора преобразования скалярного кода, включающего вызовы функций, в векторный код. Вещественный код часто содержит вызовы функций из математической библиотеки. Если библиотечная функция имеет векторы переменной длины и на входе, и на выходе, то та же функция может использоваться и для скаляров, и для векторов, а компилятор может легко векторизовать код, содержащий такие вызовы библиотечных функций.

\section{Поддержка векторов}
Векторные регистры могут содержать целые числа размером в 8, 16, 32, 64, и, возможно, 128 разрядов, или вещественные числа одинарной, двойной, и, возможно, четырёхкратной точности. Все элементы вектора обязаны иметь один и тот же тип. Элементы вектора обрабатываются параллельно. Например, векторное сложение даёт сумму двух векторов за одну операцию.

Векторные регистры имеют переменную длину. Каждый векторный регистр имеет дополнительные разряды для хранения длины вектора. Максимальная длина вектора зависит от аппаратуры. Например, если аппаратура поддерживает максимальную длину вектора, равную 64 байтам, а конкретному приложению нужно лишь 16 байт, то длина вектора устанавливается равной 16.

Некоторым командам необходимо явно указывать длину вектора, например, для чтения вектора из памяти. Эти команды используют регистр общего назначения для указания длины вектора. Длина обычно указывается как количество байтов, а не количество элементов вектора.

Сведения о максимальной длине вектора предоставляет специальный регистр. Максимальная длина, поддерживаемая процессором, должна быть степенью двойки. Используемая длина не обязана быть степенью двойки. Если используемая длина больше максимальной, то используется максимальная длина.

Содержимое векторного регистра может произвольно интерпретироваться как имеющее любые поддерживаемые типы и размеры. Например, аппаратура не предотвращает применение целочисленных команд к векторам, содержащим вещественные данные. То, что код имеет смысл --- на совести программиста.

\section{Циклы по векторам} \label{vectorLoops}
Чтобы сделать циклы по векторам более компактными, предоставляется специальный режим адресации. Он использует базовый указатель $P$ и отрицательный индекс $J$, и вычисляет адрес находящегося в памяти операнда как $P-J$, где $P$ и $J$ --- регистры общего назначения. Это делает возожным выполнение цикла по массиву так, как это показано следующим псевдокодом:
\vspace{5mm}

\fbox{\parbox{91mm}{P = адрес массива\\
J = размер массива (в байтах)\\
L = максимальная длина вектора (зависит от процессора)\\
X = векторный регистр\\
P += J; // указывает на конец массива\\
while (J > 0) \{\\
\phantom{whil}X = некая_операция(X),[P-J],(длина вектора, J)\\
\phantom{whil}J -= L;\\
\}}}
\vspace{5mm}

Данный цикл работает следующим образом: $P$ указывает на конец массива, $J$ представляет собой количество оставшихся элементов массива, уменьшающееся до тех пор, пока цикл не завершится. Цикл читает по одному вектору, находящемуся по адресу $[P-J]$, за раз. На всех итерациях цикла, кроме последней, $J$ больше максимальной длины вектора, $L$, что заставляет процессор использовать максимальную длину вектора. Если размер массива не кратен максимальной длине вектора, то на последней итерации цикл будет использовать вектор меньшей длины, чтобы поместить оставшееся количество элементов. Очевидно, что цикл может содержать любое количество операций чтения векторов, записи векторов, и векторных арифметических команд, используя тот же принцип.

Данный цикл будет работать на разных процессорах, с разны максимальными длинами векторов, \textit{не зная в момент компиляции максимальную длину вектора}. Таким образом, одна и та же часть программного обеспечения будет работать на разных микропроцессорах, с разными длинами векторов, без необходимости компилировать отдельно для каждого микропроцессора. Ещё одно преимущество состоит в том, что по завершении цикла не нужно никакого дополнительного кода для обработки оставшихся элементов, в случае, если размер массива не кратен длине вектора.

\subsubsection{Обоснование}
Большинство существующих систем имеют фиксированные длины векторов. Если разные процессоры имеют разные длины векторов, то вы должны компилировать код отдельно для каждой длины вектора. Каждый раз, когда на рынке появляется новый процессор с большей длиной векторов, вы должны откомпилировать новую версию кода, для новой длины вектора, используя вновь определённые расширения набора команд. Для нового программного обеспечения обычно требуется несколько лет для разработки и выхода на основной рынок. Для производителей программного обеспечения дорого разрабатывать и сопровождать разные версии их кода для каждой изредка появляющейся длины вектора.

Ещё одна проблема существующих систем заключается в том, что невозможно сохранить векторный регистр способом, который гарантированно совместим с последующими процессорами, имеющими более длинные вектора. Этой проблемы в дизайне ForwardCom нет, ибо длина вектора сохраняется в векторном регистре. Предоставляются команды для сохранения и восстановления векторов переменной длины и для сохранения лишь той части векторного регистра, которая действительно используется.

Дизайн ForwardCom делает возможным получение преимуществ нового процессора с более длинными векторными регистрами сразу же, без перекомпиляции кода. Метод выполнения цикла, описанный выше, делает это легко и очень эффективно. Вам  не нужны разные версии кода для разных процессоров.

Можно получить тот же самый эффект и без специального режима отрицательной адресации, если обратить знак $J$ и разрешить отрицательное значение в регистре, указывающем длину вектора, в то же время используя абсолютную величину в качестве длины вектора. Это решение менее элегантно и более запутанно, но его можно включить в дизайн ForwardCom, разрешив отрицательные значения при указании длины вектора.

Разворачивание цикла, как правило, не нужно:  накладные расходы на цикл уже уменьшены до одной команды (вычитания и перехода, если положительно), и суперскалярный процессор выполнит много итераций параллельно, если цепочки зависимости не слишком длинны. Разворачивание цикла с многими аккумуляторами может быть полезно для сокрытия проходящей через весь цикл зависимости. В этом случае вы либо вставляете в развёрнутом коде после каждой секции команды управления циклом, либо вычисляете количество итераций цикла перед его началом.

У дизайна ForwardCom нет практического ограничения на длину вектора, которую может поддерживать микропроцессор. Большай микропроцессор с очень длинными векторами может быть полезен для вычислений с высокой степенью параллелизма по данным. Обсуждались и другие способы достижения высокой производительности при параллельной обработке данных, такие, как вращающиеся стеки регистров и программная конвейеризация, но был сделан вывод, что длинные векторы --- метод, который может быть реализован эффективнее всего и в микропроцессоре, и в компиляторе.

\section{Максимальная длина вектора}
Максимальная длина векторных регистров будет разной у разных процессоров. Максимальная длина должна быть степенью двойки, и может быть столь большой, насколько желаемо, и должна быть не меньше 16 байт. Каждая команда может использовать меньшую длину, которой не нужно быть степенью двойки.

Максимальная длина может быть разной для элементов разных размеров. Например, максимальная длина для 32--разрядных целых чисел может быть равна 32 байтам, чтобы содержать восемь целых чисел, тогда как максимальная длина для 8--разрядных целых чисел могла бы быть равной 16 байтам, чтобы содержать 16 меньших чисел. Однако для разных типов с одним и тем же размером элемента максимальная длина должна быть одной и той же. Например, максимальная длина для вещественных чисел двойной точности должна быть такой же, что и для 64--разрядных целых чисел, поскольку циклы наверняка содержат оба типа, когда целочисленные вектора используются как маски для вещественных векторов. Максимальная длина для 32--разрядных элементов не может быть меньше, чем для элементов другого размера или типа операнда. Данное правило гарантирует возможность сохранения полного вектора при использовании 32--разрядного типа операнда.

Максимальная длина вектора, как правило, должна быть одна и та же для всех команд с одним и тем же типом данных. Однако могут быть исключения для команд, которые особенно дорого реализовать.

Несколько специальных регистров дают сведения о максимальной длине вектора, поддерживаемой аппаратурой, для каждого размера элемента вектора. Прикладная программа или операционная система может уменьшить максимальный размер вектора, что может быть полезным, если меньший размер вектора более подходит для конкретной цели.

Также можно уменьшить размер вектора для целей эмуляции. Виртуальные векторные регистры, которые больше, нежели поддерживает аппаратура, могут эмулироваться с помощью ловушек (синхронных прерываний), чтобы проверить функциональность программы на процессорах с большей максимальной длиной вектора, чем доступно в настоящий момент.

Когда команда указывает более длинный вектор, чем максимально возможно, то, если не активирована эмуляция больших векторов, используется максимальная длина. Это необходимо для эффективной реализации циклов по векторам, как описано выше, на с.~\pageref{vectorLoops}.

\section{Маски команд}
Большинство команд может иметь регистр маски, который может использоваться для условного выполнения и для указания различных опций. Команды, работающие с регистрами общего назначения, в качестве регистра маски или предиката используют один из регистровs r1--r7. Бит 0 регистра маски указывает, исполняется ли команда, или нет; бит 1 --- должен ли результат быть нулём или остаться неизменным, в случае, когда операция не выполняется.

Данный механизм может быть векторизован. Команды, работающие с векторными регистрами, в качестве регистров маски используют векторные регистры v1--v7. Вычисление над каждым элементом вектора обусловлено соответствующим элементом регистра маски.

Дополнительные разряды регистра маски используются для различных опций, перекрывая значения в численном управляющем регистре.

\section{Режимы адресации}
Вся адресация памяти --- относительно некоторого базового указателя. Находящийся в памяти операнд может адресоваться с помощью одной из двух общих форм:
\vspace{5mm}

\parbox{50mm}{%
Address = BP + IX*SF\\
Address = BP + OS}
\vspace{5mm}

Здесь BP --- 64--разрядный базовый указатель, IX --- 64--разрядный индексный регистр, SF --- масштабирующий множитель, а OS --- непосредственно заданное смещение. Базовый указатель присутствует всегда, остальные элементы --- необязательны.

Базовый указатель, BP, может быть регистром общего назначения, указателем секции данных (DATAP), указателем команд (IP), или указателем стека (SP).

Индексный регистр, IX, может быть одним из регистров r0--r30. Значение, равное 31, означает отсутствие индексного регистра.

К индексному регистру можно применить лимит, в виде 16--разрядного беззнакового целого числа. Если индексный регистр больше (в смысле беззнакового сравнения), чем лимит, то возбуждается синхронное прерывание (trap).

Масштабирующий множитель, SF, равен размеру (в байтах) для скалярных операндов и для размножений значений. Для векторных операндов масштабирующий множитель равен 1. Как объяснено на с.~ \pageref{vectorLoops}, также доступен специальный режим адресации с SF = $-1$.

Смещение, OS, представляет собой расширенное знаком восьми--, шестнадцати--, или тридцатидвухразрядное число. Восьмиразрядные смещения умножаются на размер операнда. У шестнадцати-- и тридцатидвухразрядных смещений множителя нет.

Поддержка режимов адресации и с индексом, и со смещением, --- необязательна.

Переходы и вызовы указывают целевой адрес относительно указателя команд. Относительный адрес указывается со знаковым смещением размером в 8, 16, 24, или 32 разряда, умноженным на размер слова кода, равный 4 байтам. При 32--разрядном смещении охватывается адресный диапазон $\pm8\text{Гб}$.


\subsubsection{Обоснование}
Используется 64--разрядное адресное пространство. Относительная адресация используется для того, чтобы в коде команды избежать 64--разрядных адресов. В том редком случае, когда 64--разрядный абсолютный адрес необходим, он должен быть загружен в регистр, который затем используется как указатель.

Адресация с индексом, масштабируемым размером операнда, полезна для массивов. К индексу может быть применён лимит, так что границы массивов можно проверить без каких--либо дополнительных команд.

Адресация с отрицательным индексом полезна для эффективной реализации циклов по векторам, описанной на с.~\pageref{vectorLoops}.

Указанные здесь режимы адресации охватят все распространённые применения, включая массивы, векторы, структуры, классы, и кадры стека.

Поддержка режима адресации с базовым указателем, индексом, и непосредственно заданным смещением, --- необязательна, ибо это потребовало бы двух сумматоров на стадии конвейера, предназначенной для вычисления адреса, что могло бы ограничить максимальную тактовую частоту.
\end{document}
