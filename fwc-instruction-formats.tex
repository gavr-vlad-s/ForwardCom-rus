% chapter included in forwardcom.tex
\documentclass[forwardcom.tex]{subfiles}
\begin{document}
%\RaggedRight

\chapter{Форматы команд}
\section{Форматы и шаблоны}
Все команды используют один из общих шаблонов форматов, показанных ниже (самые старшие разряды --- слева). Базовая компоновка 32--разрядного слова кода показана в шаблоне A. Шаблоны B, C и D получаются из шаблона A заменой 8, 16, или 24 разрядов, соответственно, непосредственно заданными константами. Команды двойного и тройного размеров можно построить, добавив к одному из этих шаблонов одного или двух 32--разрядных слов. Например, шаблон A с дополнительным 32--разрядным словом, содержащим данные, называется A2. Шаблон E2 представляет собой расширение шабона A, в котором второе слово кода содержит дополнительное регистровое поле, дополнительные разряды кода операции, разряды опций, и данные.

Некоторые малые часто используемые команды можно закодировать в малом (tiny) формате, использующем половину слова кода. Две таких малых команды можно, используя шаблон T, упаковать в одно слово кода. Неспаренная малая команда, чтобы заполнить полное слово кода, должна комбинироваться с малого размера командой NOP.

\bytefieldsetup{bitwidth=4.6mm}
\bytefieldsetup{bitheight=5mm} 
%\setlength{\bitwidth}{}
%\setlength{\byteheight}{}
\begin{figure}[!h]
\centering
{\begin{bytefield}{32}
\bitheader[b]{0,4,5,7,8,12,13,14,15,16,20,21,26,27,29,30,31}\\
\bitbox{2}{IL} & \bitbox{3}{Mode} & \bitbox{6}{OP1} & \bitbox{5}{RD} & \bitbox{1}{M} & \bitbox{2}{OT} & \bitbox{5}{RS} & \bitbox{3}{Mask} & \bitbox{5}{RT}
\end{bytefield}}
\caption{Шаблон A. Имеется три регистровых операнда и регистр маски.}\label{table:templateA}
\end{figure}

\begin{figure}[!h]
\centering
{\begin{bytefield}{32}
\bitheader[b]{0,7,8,12,13,14,15,16,20,21,26,27,29,30,31}\\
\bitbox{2}{IL} & \bitbox{3}{Mode} & \bitbox{6}{OP1} & \bitbox{5}{RD} & \bitbox{1}{M} & \bitbox{2}{OT} & \bitbox{5}{RS} & \bitbox{8}{IM1}
\end{bytefield}}
\caption{Шаблон B. Имеется два регистровых операнда и 8--разрядная непосредственно заданная константа.}\label{table:templateB}
\end{figure}

\begin{figure}[!h]
\centering
{\begin{bytefield}{32}
\bitheader[b]{0,7,8,15,16,20,21,26,27,29,30,31}\\
\bitbox{2}{IL} & \bitbox{3}{Mode} & \bitbox{6}{OP1} & \bitbox{5}{RD} & \bitbox{8}{IM2} & \bitbox{8}{IM1}
\end{bytefield}}
\caption{Шаблон C. Имеется один регистровый операнд и две 8--разрядных непосредственно заданных константы.}\label{table:templateC}
\end{figure}

\begin{figure}[!h]
\centering
{\begin{bytefield}{32}
\bitheader[b]{0,23,24,26,27,29,30,31}\\
\bitbox{2}{IL} & \bitbox{3}{Mode} & \bitbox{3}{OP1} & \bitbox{24}{IM2} 
\end{bytefield}}
\caption{Шаблон D. Нет регистрового операнда, но есть непосредственно заданная 24--разрядная константа.}\label{table:templateD}
\end{figure}

\begin{figure}[!h]
\centering
{\begin{bytefield}{32}
\bitheader[b]{0,4,5,7,8,12,13,14,15,16,20,21,26,27,29,30,31}\\
\bitbox{2}{IL} & \bitbox{3}{Mode} & \bitbox{6}{OP1} & \bitbox{5}{RD} & \bitbox{1}{M} & \bitbox{2}{OT} & \bitbox{5}{RS} & \bitbox{3}{Mask} & \bitbox{5}{RT}\\
\bitbox{5}{OP2}                   & \bitbox{6}{OP3} & \bitbox{5}{RU} & \bitbox{16}{IM2}                                                                   
\end{bytefield}}
\caption{Шаблон E2. Имеется 4 регистровых операнда, маска, 16--разрядная непос\-редственно заданная константа, и дополнительные разряды для кода операции или опций.}\label{table:templateE2}
\end{figure}

\newpage

\begin{figure}[!h]
\centering
{\begin{bytefield}{32}
\bitheader[b]{0,4,5,7,8,12,13,14,15,16,20,21,26,27,29,30,31}\\
\bitbox{2}{IL} & \bitbox{3}{Mode} & \bitbox{6}{OP1} & \bitbox{5}{RD} & \bitbox{1}{M} & \bitbox{2}{OT} & \bitbox{5}{RS} & \bitbox{3}{Mask} & \bitbox{5}{RT} \\
\wordbox{1}{IM2}                                                                                                                                           
\end{bytefield}}
\caption{Шаблон A2. Два слова. Как A, но с дополнительной 32--разрядной непос\-редственно заданной константой.}\label{table:templateA2}
\end{figure}

\begin{figure}[!h]
\centering
{\begin{bytefield}{32}
\bitheader[b]{0,4,5,7,8,12,13,14,15,16,20,21,26,27,29,30,31}\\
\bitbox{2}{IL} & \bitbox{3}{Mode} & \bitbox{6}{OP1} & \bitbox{5}{RD} & \bitbox{1}{M} & \bitbox{2}{OT} & \bitbox{5}{RS} & \bitbox{3}{Mask} & \bitbox{5}{RT} \\
\wordbox{1}{IM2}                                                                                                                                           \\
\wordbox{1}{IM3}
\end{bytefield}}
\caption{Шаблон A3. Три слова. Как A, но с двумя дополнительными 32-разрядными непос\-редственно заданными константами.}\label{table:templateA3}
\end{figure}

\begin{figure}[!h]
\centering
{\begin{bytefield}{32}
\bitheader[b]{0,7,8,12,13,14,15,16,20,21,26,27,29,30,31}\\
\bitbox{2}{IL} & \bitbox{3}{Mode} & \bitbox{6}{OP1} & \bitbox{5}{RD} & \bitbox{1}{M} & \bitbox{2}{OT} & \bitbox{5}{RS} & \bitbox{8}{IM1} \\
\wordbox{1}{IM2}                                                                                                                                          
\end{bytefield}}
\caption{Шаблон B2. Как B, но с дополнительной 32--разрядной непос\-редственно заданной константой.}\label{table:templateB2}
\end{figure}

\begin{figure}[!h]
\centering
{\begin{bytefield}{32}
\bitheader[b]{0,7,8,12,13,14,15,16,20,21,26,27,29,30,31}\\
\bitbox{2}{IL} & \bitbox{3}{Mode} & \bitbox{6}{OP1} & \bitbox{5}{RD} & \bitbox{1}{M} & \bitbox{2}{OT} & \bitbox{5}{RS} & \bitbox{8}{IM1} \\
\wordbox{1}{IM2}                                                                                                                         \\
\wordbox{1}{IM3}                 
\end{bytefield}}
\caption{Шаблон B3. Как B, но с двумя дополнительными 32-разрядными непос\-редственно заданными константами.}\label{table:templateB3}
\end{figure}

\begin{figure}[!h]
\centering
{\begin{bytefield}{32}
\bitheader[b]{0,7,8,15,16,20,21,26,27,29,30,31}\\
\bitbox{2}{IL} & \bitbox{3}{Mode} & \bitbox{6}{OP1} & \bitbox{5}{RD} & \bitbox{8}{IM2} & \bitbox{8}{IM1} \\
\wordbox{1}{IM3}                 
\end{bytefield}}
\caption{Шаблон C2. Как C, но с дополнительной 32--разрядной непос\-редственно заданной константой.}\label{table:templateC2}
\end{figure}

\begin{figure}[!h]
\centering
{\begin{bytefield}{32}
\bitheader[b]{0,13,14,27,28,31}\\
\bitbox{4}{0111} & \bitbox{14}{\text{Малая команда 2}} & \bitbox{14}{\text{Малая команда 1}}
\end{bytefield}}
\caption{Шаблон T. Одно слово, содержащее две малых команды.}\label{table:templateT}
\end{figure}

\begin{figure}[!h]
\centering
{\begin{bytefield}{14}
\bitheader[b]{0,3,4,8,9,13}\\
\bitbox{5}{OP1} & \bitbox{5}{RD} & \bitbox{4}{RS}
\end{bytefield}}
\caption{Формат каждой малой команды.}\label{table:tinyFormat}
\end{figure}

Смысл каждого поля описан в следующей таблице.

\begin{longtable} {|p{16mm}|p{32mm}|p{101mm}|}
\caption{Поля в шаблоне команды} \label{table:fieldsInTemplates} \\
\endfirsthead
\endhead
\hline
Имя поля & Смысл & Значения  \\
\hline
IL & Длина команды & 0 или 1: 1 слово = 32 разряда \newline
2: 2 слова = 64 разряда \newline
3: 3 или более слов  \\
\hline
Mode & Формат & Определяет формат шаблона и использование каждого поля.
Ког\-да необходимо, расширяется разрядами M. Детали см. ниже. \\ \hline
OP1 & Код операции & Определяет операцию, например, сложение или пересылку.  \\ \hline
OT & Тип и размер\newline (OS) операнда & 
0: 8--разрядное целое, OS = 1 байт  \newline
1: 16--разрядное целое, OS = 2 байта \newline
2: 32--разрядное целое, OS = 4 байта \newline
3: 64--разрядное целое, OS = 8 байт \newline
4: 128--разрядное целое, OS = 16 байт (необязательно) \newline
5: вещественное одинарной точности, OS = 4 байта \newline
6: вещественное двойной точности, OS = 8 байт \newline
7: вещественное четырёхкратной точности, OS = 16 байт (необязательно) \newline
Когда необходимо, поле OT расширяется разрядами M. \\ \hline
RD & Регистр--при\-ём\-ник & r0--r31 или v0--v31. Также используется для первого операнда--источника, если формат команды не указывает достаточное количество операндов. \\ \hline
RS & Регистр--источник & r0--r31 или v0--v31. Регистр--источник, указатель, индекс, или регистр длины вектора. \\ \hline
RT & Регистр--источник & r0--r31 или v0--v31. Регистр--источник или указатель.  \\ \hline
RU & Регистр--источник & r0--r31 или v0--v31. Регистр--источник. \\ \hline
Mask & Регистр маски & 0 означает отсутствие маски. 1--7 означает, что для маски и битов опций используется регистр общего назначения или векторный регистр.  \\ \hline
M & Тип операнда или\newline режим. & Расширяет поле Mode, когда разряды 1 и 2 этого поля оба равны нулю (регистры общего назначения). В противном случае расширяет поле OT (векторные регистры).  \\ \hline
OP2 & Код операции & Расширение кода операции. \\ \hline
IM1, IM2, IM3 & Непосредственно заданные константы & 8--, 16--, 32--, или 64--разрядный непосредственно заданный операнд, или адресное смещение, или разряды опций. Соседние поля IM могут быть слиты вместе. \\ \hline
OP3 & Опции & Разряды опций, разряды режима, или непосредственно заданные константы. \\ \hline
\end{longtable}

Согласно приводимой ниже таблице, у команд есть несколько различных форматов, определяемых полем IL и разрядами режима.

\begin{longtable} {|p{15mm}|p{9mm}|p{9mm}|p{13mm}|p{95mm}|}
\caption{Список форматов команд} \label{table:instructionFormats} \\
\endfirsthead
\endhead
\hline
Имя формата & IL & Mode & Шаблон & Использование  \\
\hline
0.0 & 0 & 0 & A & Три операнда (RD, RS, RT), являющихся регистрами общего наз\-начения.  \\ \hline
0.1 & 0 & 1 & B & Два регистра общего назначения (RD, RS) и 8--разрядный непосредственный операнд (IM1). \\ \hline
0.2 & 0 & 2 & A & Три операнда (RD, RS, RT), являющихся векторными регистрами. \\ \hline
0.3 & 0 & 3 & B & Два векторных регистра (RD, RS) и размножаемый 8--разрядный непосредственный операнд (IM1). \\ \hline
0.4 & 0 & 4 & A & Один векторный регистр (RD), находящийся в памяти операнд с указателем (RT) и длина вектора, указанная в регистре общего назначения (RS). \\ \hline
0.5 & 0 & 5 & A & Один векторный регистр (RD) и находящийся в памяти операнд с базовым указателем (RT). Отрицательный индекс и длина вектора указаны в RS. Используется для циклов по векторам, как объяснено на с.~\pageref{vectorLoops}. \\ \hline
0.6 & 0 & 6 & A & Один векторный регистр (RD) и находящийся в памяти скалярный операнд, имеющий базовый указатель (RT) и индекс (RS), умножаемый на размер операнда. \\ \hline
0.7 & 0 & 7 & B & Один векторный регистр (RD) и находящийся в памяти скалярный операнд, имеющий базовый указатель (RS) и 8--разрядное смещение. \\ \hline
0.8 & 0 & 0 M=1 & A & Один регистр общего назначения (RD) и находящийся в памяти операнд, имеющий базовый указатель (RT) и индекс (RS), умножаемый на размер операнда.  \\ \hline
0.9 & 0 & 1 M=1 & B & Один регистр общего назначения (RD) и находящийся в памяти операнд, имеющий базовый указатель (RT) и 8--разрядное смещение. \\ \hline
1.0 & 1 & 0 & A & Одноформатные команды. Три регистра общего назначения в качестве операндов. \\ \hline
1.1 & 1 & 1 & C & Одноформатные команды. Один регистр общего назначения и 16--разрядный непосредственный операнд. \\ \hline
1.2 & 1 & 2 & A & Одноформатные команды. Три векторных регистра в качестве операндов.  \\ \hline
1.3 & 1 & 3 & B, C & Одноформатные команды. Два векторных регистра и размножаемый 8--разрядный непосредственный операнд, либо один векторный регистр и размножаемый 16--разрядный операнд. \\ \hline
1.4 & 1 & 4 & B & Команды перехода с двумя регистровыми операндами и 8--разрядным смещением.  \\ \hline
1.5 & 1 & 5 & C, D & Команды перехода с одним регистровым операндом, 8--разрядной константой (IM2), и 8--разрядным смещением (IM1); либо нет регистрового операнда, но есть 24--разрядное смещение.  \\ \hline
1.8 & 1 & 0 M=1 & B & Одноформатные команды. Два регистра общего назначения и 8--разрядный непосредственный операнд. \\ \hline
T & 1 & 6-7 & T & Две малых команды. \\ \hline
2.0 & 2 & 0 & A2 & Два регистра общего назначения (RD, RS) и находящийся в памяти операнд с базовым указателем (RT) и 32--разрядным смещением (IM2). \\ \hline
2.1 & 2 & 1 & A2 & Три регистра общего назначения и 32--разрядный непосредственный операнд IM2. \\ \hline
2.2 & 2 & 2 & A2 & Один векторный регистр (RD) и находящийся в памяти операнд с базовым указателем (RT) и 32--разрядным смещением (IM2). Длина вектора указывается регистром общего назначения RS. \\ \hline
2.3 & 2 & 3 & A2 & Три вектоных регистра и размножаемый 32--разрядный непосредственный операнд IM2. \\ \hline
2.4.0 & 2 & 4 & E2 & OP3=00xxxx. Два векторных регистра (RD, RU) и находящийся в памяти скалярный операнд с базовым регистром (RT) и 16--разрядным смещением (IM2), расширяемым до длины (RS). \\ \hline
2.4.1 & 2 & 4 & E2 & OP3=01xxxx. Два векторных регистра (RD, RU) и находящийся в памяти скалярный операнд с базовым регистром (RT), 16--разрядным смещением (IM2), длиной (RS). \\ \hline
2.4.2 & 2 & 4 & E2 & OP3=10xxxx. Два векторных регистра (RD, RU) и находящийся в памяти операнд с базовым регистром (RT), отрицательным индексом (RS), and length (RS). Optional support for offset IM2 $\neq$ 0, otherwise IM2 = 0. \\ \hline
2.4.3 & 2 & 4 & E2 & OP3=11xxxx. Два векторных регистра (RD, RU) и находящийся в памяти скалярный операнд с базовым регистром (RT), масштабируемый индекс (RS), и лимит RS $\leq$ IM2 (беззнаково). \\ \hline
2.5 & 2 & 5 & E2 & Три вектоных регистра (RD, RS, RT) и размножаемое 16--разрядное непосредственно заданное целое число IM2. IM2 сдвигается влево на значение, указанное 6--разрядной беззнаковой величиной OP3, если OP3 не используется для других целей. RU обычно не используется. \\ \hline
2.6 & 2 & 6 & A2 & Одноформатные команды. Три регистра общего назначения и 32--разрядный непосредственный операнд. \\ \hline
2.7 & 2 & 7 & A2, B2, C2 & Команды перехода (OP1 $<$ 16). Одноформатные команды. Три векторных регистра и 32--разрядный непосредственный операнд. \\ \hline
2.8.0 & 2 & 0 M=1 & E2 & OP3=00xxxx. Три регистра общего назначения (RD, RS, RU) и находящийся в памяти операнд с базовым регистром (RT) и 16--разрядным смещением (IM2).  \\ \hline
2.8.1 & 2 & 0 M=1 & E2 & OP3=01xxxx. Два регистра общего назначения (RD, RU) и находящийся в памяти операнд с базовым регистром (RT), индексом (RS), и без масштабирования. Необязательная поддержка для смещения IM2 $\neq$ 0, иначе IM2 = 0. \\ \hline
2.8.2 & 2 & 0 M=1 & E2 & OP3=10xxxx. Два регистра общего назначения (RD, RU) и находящийся в памяти операнд с базовым регистром (RT) и масштабируемым индексом (RS). Необязательная поддержка для смещения IM2 $\neq$ 0, иначе IM2 = 0. \\ \hline
2.8.3 & 2 & 0 M=1 & E2 & OP3=11xxxx. Два регистра общего назначения (RD, RU) и находящийся в памяти операнд с базовым регистром (RT), масштабируемым индексом (RS), и лимит RS $\leq$ IM2 (беззнаково). \\ \hline
2.9 & 2 & 1 M=1 & E2 & Три регистра общего назначения (RD, RS, RT) и 16--разрядное непосредственно заданное целое число IM2. IM2 сдвигается влево на значение, указанное 6--разрядной беззнаковой величиной OP3, если OP3 не используется для других целей. RU обычно не используется. \\ \hline
3.0 & 3 & 0 & A3, B3 & Команды перехода. Одноформатные команды с регистрами общего назначения в качестве операндов. Необязательно. \\ \hline
3.1 & 3 & 1 & A3 & Три регистра общего назначения и 64--разрядный непосредственно заданный операнд. Необязательно.  \\ \hline
3.2 & 3 & 2 & A3 & Одноформатные векторные команды. Необязательно. \\ \hline
3.3 & 3 & 3 & A3 & Три векторных регистра и размножаемый 64--разрядный непосредственно заданный операнд. Необязательно. \\ \hline
3.8 & 3 & 0 M=1 & & В настоящее время не использется. \\ \hline
4.x & 3 & 4-7 &  & Зарезервировано для последующих команд, размером в 4 и более слова. \\ \hline
\end{longtable}


\section{Кодирование операндов}
\subsection{Тип операнда}
Тип и размер операндов определяется полем OT, как указано выше. Если поля OT нет, то, по умолчанию, тип операнда --- 64--разрядное целое (OS = 8). 


\subsection{Тип регистра}
Команды могут использовать либо регистры общего назначения, либо векторные регистры. Регистры общего назначения используются и для операндов--источников, и для операндов--приёмников, и для маск, если режим равен 0 или 1 (с M = 0 или 1). Векторные регистры используются и для операндов--источников, и для операндов--приёмников, и для маск, если режим равен значению между 2 и 7. Равное нулю значение поля маски означает, что маски нет, и операция является безусловной.

\subsection{Регистры--указатели}
Команды с находящимся в памяти операндом всегда используют адреса относительно базового указателя. Базовый указатель может быть регистром общего назначения, указателем секции данных, или указателем команд. Базовый указатель определяется полем RS или RT. Это поле интерпретируется следующим образом.

Команды с форматами без смещения или с 8--разрядным смещением (0.4-0.9) могут использовать любой из регистров r0--r31 в качестве базового указателя. Регистр r31 является указателем стека.

Команды с форматами, имеющими 16--разрядное или 32--разрядное смещение (2.0, 2.2, 2.4, 2.8), могут использовать те же регистры, кроме r29, заменяемого указателем секции данных (DATAP), и r30, заменяемого указателем команд (IP). Это применимо также и к форматам с неиспользуемым 16--разрядным смещением (форматы 2.4.2 и 2.4.3).

Малые команды, имеющие находящийся в памяти операнд, в качестве указателя в 4--разрядном RS поле могут использовать r0--r14 или указатель стека (r31) --- равное 15 значение поля RS обозначает указатель стека.

\subsection{Индексные регистры}
Команды с форматами, имеющимм индекс, в качестве индекса могут использовать r0--r30. Значение в поле индекса (RS), равное 31, означает отсутствие индекса. Знаковый индекс умножается на размер операнда (OS) для форматов 0.6, 0.8, 2.4.3, 2.8.2, 2.8.3; на 1 для формата 2.8.1; или -1 для форматов 0.5 и 2.4.2. Результат складывается со значением базового указателя.

\subsection{Смещения}
Смещения могут быть восьми--, шестнадцати--, или тридцатидвухразрядными. Значение смещения расширяется знаком до 64 разрядов. Восьмиразрядное смещение умножается на размер операнда (OS), определяемый полем OT. Шестнадцатиразрядное или тридцатидвухразрядное смещение не масштабируется. Результат складывается со значением базового указателя.

Поддержка режимов адресации и с индексом, и со смещением (форматы  2.4.2, 2.8.1, 2.8.3) --- необязательна. Если этот тип адресации, требующий двух сложений, не поддерживается, то смещение в IM2 должно быть нулём.

\subsection{Лимит для индекса}
В форматах 2.4.3 и 2.8.3 имеется 16--разрядный лимит для индексного регистра, что полезно для проверки границ массивов. Если значение индексного регистра, рассматриваемое как беззнаковое целое число, больше беззнакового лимита, то порождается ловушка (trap, синхронное прерывание).

\subsection{Длина вектора}
Длина находящегося в памяти вектора указывается (для форматов 0.4, 0.5, 2.2, 2.4) в поле RS, регистрами r0--r30. Значение поля RS, равное 31, используется для обозначения скаляра той же длины, что и размер операнда (OS).

Значение длины находящегося в регистре вектора задаёт длину векторного операнда, находящегося в памяти, в байтах, а не в количестве элементов. Если это значение больше максимально возможной длины вектора, то используется максимальная длина вектора. Длина вектора может быть равна нулю. Поведение для отрицательных значений длины зависит от реализации: либо данное значение рассматривается как беззнаковое, либо используется абсолютная величина.

Длина вектора должна быть кратна размеру операнда (OS), указываемому полем OT. Если длина вектора не кратна размеру операнда, то поведение для частично определённого элемента вектора зависит от реализации.

Длина вектора для находящихся в векторных регистрах операндов--источников сохраняется в регистре.

\subsection{Комбинирование векторов с различными длинами}
Длина вектора, находящегося в приёмнике, будет такой же, что  длина вектора, находящегося в первом из операндов--источников, даже если первый операнд--источник использует поле RD.

Как следствие, при комбинировании векторов с разными длинами длина результата определяется порядком операндов.

Если операнды--источники имеют разные длины, то длины будут настроены так, как показано ниже. Если векторный операнд--источник слишком длинен, то лишние элементы игнорируются. Если векторный операнд--источник слишком короток, то отсутствующие элементы будут равны нулю.

Находящийся в памяти скалярный операнд (форматы 0.6 и 0.7) не размножается, а рассматривается как короткий вектор, и дополняется нулями до длины вектора--приёмника.

Размножаемый операнд, находящийся в памяти (формат 2.4.1) будет использовать длину вектора, заданную регистром, указанным в поле RS.

Размножаемый непосредственно заданный операнд будет использовать ту же длину вектора, что и операнд--приёмник.

\subsection{Непосредственно заданные константы}
Непосредственно заданные константы могут иметь разрядность, равную 4, 8, 16, 32, и, необязательно, 64. Непосредственно заданные поля, как правило, выровнены на естественные адреса, и интерпретируются следующим образом.

Если поле OT указывает целочисленный тип, то поле (непосредственно заданное) рассматривается как целое число. Если поле меньше размера операнда, то оно расширяется знаком до подходящего размера. Если поле больше размера операнда, то излишние разряды игнорируются. Усечение слишком большого непосредственно заданного операнда не взведёт никакого условия переполнения.

Если поле OT указывает вещественный тип, то поле рассматривается так. Непосредственно заданные поля разрядности, меньшей 32, интерпретируются как знаковые целые числа, и преобразуются в вещественные числа желаемой точности. Тридцатидвухразрядное поле рассматривается как вещественное число одинарной точности, и, если необходимо, преобразуется к желаемой точности. Шестидесятичетырёхразрядное поле (если таковое поддерживается) интерпретируется как вещественное число двойной точности. Шестидесятичетырёхразрядное поле для типа операнда, являющегося числом одинарной точности, не допускается. Несколько необязательных команд формата 1.3C имеют в качестве операндов вещественные непосредственно заданные константы половинной точности, которые преобразуются в скаляр одинарной или двойной точности.

Шестнадцатиразрядные константы в форматах 2.5 и 2.9 могут сдвигаться влево на значение, указанное 6--разрядной беззнаковой величиной OP3, чтобы получить 64--разрядное знаковое значение. Всё, что окажется за пределами 64 разрядов, игнорируется. Сдвиг выполняется до какого--либо преобразования в вещественные числа. Никакого сдвига не выполняется, если OP3 используется в других целях.

Команду можно сделать компактной, если использовать наименьший размер поля непосредственно заданного операнда, в который помещается действительное значение константы.

\subsection{Регистры масок}
Трёхразрядное поле маски указывает регистр маски. Если приёмником является регистр общего назначения, то используются регистры r1--r7; а если приёмник --- векторный регистр, то используются регистры v1--v7. Значение поля маски, равное нулю, означает отсутствие маски и безусловное выполнение, с использованием опций, указанных в численном управляющем регистре.

Если маска является векторным регистром, то она рассматривается как вектор, имеющий тот же размер элемента, что и указанный полем OT. Каждый элемент регистра-- маски применяется к соответствующим компонентам результата.

Смысл регистров флагов описан в следующем разделе.

\section{Кодирование масок}
Регистр маски может быть регистром общего назначения r1--r7 или векторным регистром v1--v7. Равное нулю значение поля маски означает отсутствие маски.

Разряды регистра маски кодируются так.

\newpage

\begin{longtable}{|p{15mm}|p{135mm}|}
\caption{Разряды регистра маски и численного управляющего регистра}
\label{table:maskBits}
\endfirsthead
\endhead
\hline
Номер разряда & Смысл \\
 \hline
0 & Предикат или маска. Операция выполняется только в том случае, если этот разряд равен единице. Если этот разряд равен нулю, то операция не выполняется, и подавляются все условия проверки арифметических ошибок. \\ \hline
1 & Зануление. Этот разряд определяет результат, когда разряд №0 равен 0. Равенство нулю этого разряда зануляет результат, а равенство единице оставляет значение без изменения, т.е. результат --- такой же, как и значение в первом операнде--источнике на входе. Разряд №1 не оказывает влияния, когда разряд №0 равен единице. \\ \hline
2 & Обнаруживать беззнаковое целочисленное переполнение. \\ \hline
3 & Обнаруживать знаковое целочисленное переполнение. \\ \hline
6 & Распространять ошибочные разряды, обнаруженные разрядами №№2 и 3. Это экспериментальная возможность, см. с.~\pageref{integerOverflowDetection}. \\ \hline
7 & Возбудить прерывание, если обнаружено переполнение, указанное разрядами №2 или 3. \\ \hline
18-19 & Режим округления для вещественных чисел: \newline
00 = к ближайшему или чётному \newline
01 = в меньшую сторону \newline
10 = в большую сторону \newline
11 = к нулю \\ \hline
20 & Поддерживать денормализованные числа. Денормализованные вещественные числа рассматриваются как нуль (это, как правило, быстрее), когда данный разряд равен 0. \\ \hline
22 & Более хорошее распространение нечисел (NAN). Если равен нулю этот разряд, то строго придерживаться стандарта IEEE 754-2008 (или более позднего) для значений NAN. Равное единице значение разряда №22 улучшает распространение NAN и использует значения NAN для отслеживания ошибок вычислений с плавающей запятой. Детали описаны на с.~\pageref{nanPropagation}. \\ \hline
26 & Разрешить возбуждение прерывания при вещественном переполнении и при делении на нуль. \\ \hline
27 & Разрешить возбуждение прерывания при выполнении недопустимой операции с плавающей запятой. \\ \hline
28 & Разрешить возбуждение прерывания при вещественном антипереполнении и потере точности. \\ \hline
29 & Разрешить возбуждение прерывания при NAN в качестве входных аргументов команд сравнения и команд преобразования вещественных чисел в целые.  \\ \hline
\end{longtable}

Разряды 8--9, 16--17, 24--25, и т.д. в векторном регистре маски могут использоваться подобно разрядам 0--1 для 8--разрядных и 16--разрядных операндов. Все прочие разряды зарезервированы для использования в будущем.

Векторные команды трактуют регистр маски как вектор с тем же размером элемента (OS), что и у операндов. У каждого элемента вектора--маски имеются битовые коды, перечисленные выше. У разных элементов вектора могут быть разные разряды маски.

Когда поле маски равно нулю или отсутствует, в качестве маски используется численный управляющий регистр (NUMCONTR). Когда у команды нет регистра маски, регистр NUMCONTR размножается для всех элементов вектора, используя столько разрядов регистра NUMCONTR, сколько указано размером операнда. В этом случае ко всем элементам вектора применяется одна и та же маска. Разряд №0 регистра NUMCONTR обязан быть равным 1.

\section{Формат команд перехода, вызова, и ветвления}
Большинство ветвлений в обычном коде основаны на результате арифметической или логической команды (АЛУ). Дизайн ForwardCom комбинирует команду АЛУ и условный переход в одну команду. Например, цикл можно реализовать одной командой, которая уменьшает счётчик и выполняет переход, если счётчик не достиг нуля, либо увеличивает счётчик с отрицательного значения, и выполняет переход, если это значение не достигло нуля.

Переходы, вызовы, ветвления, и многопутёвые ветвления используют следующие форматы.

\begin{longtable}
{|p{14mm}|p{9mm}|p{9mm}|p{9mm}|p{14mm}|p{84mm}|}
\caption{Список форматов для команд передачи управления}
\label{table:jumpInstructionFormats}
\endfirsthead
\endhead
\hline
Формат & IL & Mode & OP1 & Шаблон & Описание \\ \hline
1.4    & 1  & 4    & OPJ & B      & Короткая версия, с двумя регистровыми операндами (RD, RS) и 8--разрядным смещением (IM1).  \\ \hline
1.5 C  & 1  & 5    & OPJ & C      & Короткая версия, с одним регистровым операндом (RD), и либо с 8--разрядной непосредственно заданной константой (IM2) и 8--разрядным смещением (IM1), либо с 16--разряднымым смещением (склеены IM2 и IM1). \\ \hline
1.5 D  & 1  & 5    & 0-7 & D      & Переход или вызов с 24--разрядным смещением. \\ \hline
2.7.0  & 2  & 7    & 0   & B2     & Версия двойного размера, с двумя регистровыми операндами и с 32--разрядным смещением (IM2). IM1 = OPJ. \\ \hline
2.7.1  & 2  & 7    & 1   & B2     & Версия двойного размера, с регистровым операндом--приёмником, регистровым операндом--источником, 16--разрядным смещением (младшая половина IM1) и 16--разрядным непосредственно заданным операндом (старшая половина IM2).  \\ \hline
2.7.2  & 2  & 7    & 2   & C2     & Версия двойного размера, с одним регистровым операндом (RD), одной 8--разрядной непосредственно заданной константой (IM2) и 32--разрядным смещением (IM3). \\ \hline
2.7.3  & 2  & 7    & 3   & C2     & Версия двойного размера, с одним регистровым операндом (RD), 8--разрядным смещением (IM2) и 32--разрядной непосредственно заданной константой (IM3).  \\ \hline
2.7.4  & 2  & 7    & 4   & C2     & Двойного размера системный вызов, без OPJ, с 16--разрядной константой (IM1,IM2), и с 32--разрядной константой (IM3). \\ \hline
3.0.0  & 3  & 0    & 0   & C2     & Нет операции (NOP). \\ \hline
3.0.1  & 3  & 0    & 1   & B3     & Версия тройного размера, с регистровым операндом--приёмником, регистровым операндом--источником, 32--разрядным непосредственно заданным операндом (IM2), и 32--разрядным смещением (IM3). Необязательно.  \\ \hline
\end{longtable}

Команды перехода, вызова, и ветвления имеют знаковые смещения размером в 8, 16, 24, или 32 разряда, относительно указателя команд. Или, точнее, относительно конца команды. Данное смещение умножается на размер слова команды (равный 4 байтам), чтобы охватить диапазон в плюс--минус полкилобайта для коротких условных переходов с 8--разрядным смещением, в плюс--минус 32 мегабайта для 24--разрядных смещений, и плюс--минус 8 гигабайт для 32--разрядных смещений. Необязательный формат тройного размера включает безусловный переход и вызов с 64--разрядным абсолютным адресом.

Версия с шаблонами C и C2 не имеет поля OT. Когда поля OT нет, тип операнда --- 64--разрядное целое. С вещественными типами использовать шаблоны C и C2 невозможно. Когда есть поле OT и M=1, команды будут использовать векторные регистры (только первый элемент). Иными словами, команды АЛУ--перехода будут использовать векторные регистры только когда указан вещественный тип (или, если таковой поддерживается, 128--разрядный целочисленный тип). Во всех иных случаях используются регистры общего назначения. Можно использовать поразрядные логические команды с векторными командами, указав вещественный тип.

Поле OPJ определяет операцию и условие перехода. Данное поле --- 6--разрядное в версии одинарного размера, и 8--разрядное --- в более длинных версиях. Два дополнительных разряда в более длинных версиях используются так: бит 6 зарезервирован для использования в будущем, и обязан быть равным нулю; а бит 7 может использоваться для указания поведения цикла, как подсказка для выбора оптимальной ветви алгоритмом предсказания ветвлений.

Младшие 6 разрядов поля OPJ содержат следующие коды:

\begin{longtable}{|p{10mm}|p{18mm}|p{80mm}|p{35mm}|}
%\nopagebreak
\caption{Список команд передачи управления: переходы, вызовы, возвраты}
\label{table:controlTransferInstructions}
\endfirsthead
\endhead
\hline
OPJ   & Бит 0 поля OPJ & Функция                                                           & Комментарий   \\ \hline
0-7   & часть смещения & Безусловный переход с 24--разрядным смещением.                    & Формат 1.5 D. \\ \hline
8-15  & часть смещения & Безусловный вызов с 24--разрядным смещением.                      & Формат 1.5 D. \\ \hline
0-1   & ин\-вер\-ти\-ро\-ван   & Знаково вычесть, и перейти, если отрицательно\newline (sub\_sign\_jmpneg).    & Форматы 1.4 и 2.7.0. Не для плавающей запятой. \\ \hline
2-3   & ин\-вер\-ти\-ро\-ван   & Знаково вычесть, и перейти, если положительно\newline (sub\_sign\_jmppos).    & Форматы 1.4 и 2.7.0. Не для плавающей запятой. \\ \hline
4-5   & ин\-вер\-ти\-ро\-ван   & Беззнаково вычесть, и перейти, если заём\newline (sub\_unsign\_jmpborrow).    & Форматы 1.4 и 2.7.0. Не для плавающей запятой. \\ \hline
6-7   & ин\-вер\-ти\-ро\-ван    & Беззнаково вычесть, и перейти, если не нуль либо заём (sub\_unsign\_jmpnzc). & Форматы 1.4 и 2.7.0. Не для плавающей запятой. \\ \hline
8-9   & ин\-вер\-ти\-ро\-ван    & Вычесть, и перейти, если не ноль (sub\_jmpnzero).                            &  Форматы 1.4 и 2.7.0. Не для плавающей запятой. \\ \hline
10-11 & ин\-вер\-ти\-ро\-ван    & Знаково вычесть, и перейти, если переполнение (sub\_sign\_jmpovfl).          & Форматы 1.4 и 2.7.0. Не для плавающей запятой. \\ \hline
12-15 & & Зарезервировано для последующего использования. & Форматы 1.4 и 2.7.0. \\ \hline
16-17 & ин\-вер\-ти\-ро\-ван    & Знаково сложить, и перейти, если отрицательно (add\_sign\_jmpneg).           & Не для плавающей запятой. \\ \hline
18-19 & ин\-вер\-ти\-ро\-ван    & Знаково сложить, и перейти, если положительно (add\_sign\_jmppos).           & Не для плавающей запятой. \\ \hline
20-21 & ин\-вер\-ти\-ро\-ван    & Беззнаково сложить, и перейти, если перенос (add\_unsign\_jmpcarry).         & Не для плавающей запятой. \\ \hline
20-21 & ин\-вер\-ти\-ро\-ван    & Перейти, если один из операндов равен $\pm\infty$ или является NAN (cmp\_float\_jmpinfnan). & Для вещественных операндов. \\ \hline
22-23 & ин\-вер\-ти\-ро\-ван    & Беззнаково сложить, и перейти, если не нуль, либо перенос (add\_unsign\_jmpnzc). & Не для плавающей запятой. \\ \hline
22-23 & ин\-вер\-ти\-ро\-ван    & Перейти, если один из операндов --- денормализован (cmp\_float\_jmpsubnorm). & Для вещественных операндов. \\ \hline
24-25 & ин\-вер\-ти\-ро\-ван    & Сложить, и перейти, если не нуль (add\_jmpnzero).                            & Не для плавающей запятой. \\ \hline
26-27 & ин\-вер\-ти\-ро\-ван    & Знаково сложить, и перейти, если переполнение (add\_sign\_jmpovfl).          & Не для плавающей запятой.  \\ \hline
28-29 &  ин\-вер\-ти\-ро\-ван   & Сдвинуть влево на n разрядов, и перейти, если не нуль (shift\_jmpnzero).     & Беззнаково cдвинуть вправо, если n отрицательно. \\ \hline
30-31 & ин\-вер\-ти\-ро\-ван    & Сдвинуть влево на n разрядов, и перейти, если перенос (shift\_jmpcarry).     & Беззнаково cдвинуть вправо, если n отрицательно.  \\ \hline
32-33 & ин\-вер\-ти\-ро\-ван    & Знаково сравнить, и перейти, если ниже\newline (cmp\_sign\_jmpbelow).      &  \\ \hline
34-35 & ин\-вер\-ти\-ро\-ван    & Знаково сравнить, и перейти, если выше\newline (cmp\_sign\_jmpabove).      &  \\ \hline
36-37 & ин\-вер\-ти\-ро\-ван    & Беззнаково сравнить, и перейти, если ниже\newline (cmp\_unsign\_jmpbelow). & Целочисленные операнды. \\ \hline
36-37 & ин\-вер\-ти\-ро\-ван    & Перейти, если один из операндов --- NAN (cmp\_float\_jmpunordered).        & Вещественные операнды. \\ \hline
38-39 & ин\-вер\-ти\-ро\-ван    & Беззнаково сравнить, и перейти, если выше\newline (cmp\_unsign\_jmpabove). & Целочисленные операнды.  \\ \hline
38-39 & ин\-вер\-ти\-ро\-ван    & Перейти, если один из операндов равен $\pm\infty$ (cmp\_float\_jmpinf).    & Вещественные операнды. \\ \hline
40-41 & ин\-вер\-ти\-ро\-ван    & Сравнить, и перейти, если не равно (cmp\_jmpneq). & \\ \hline
42-43 & ин\-вер\-ти\-ро\-ван    & Поразрядное И без записи результата, и перейти, если не ноль (test\_jmpnzero). &  \\ \hline
44-45 & ин\-вер\-ти\-ро\-ван    & Поразрядное И, и перейти, если не ноль\newline (and\_jmpnzero). & \\ \hline
46-47 & ин\-вер\-ти\-ро\-ван    & Поразрядное или, и перейти, если не ноль\newline (or\_jmpnzero). &  \\ \hline
48-49 & ин\-вер\-ти\-ро\-ван    & Поразрядное ИСКЛЮЧАЮЩЕЕ ИЛИ, и перейти, если не ноль (xor\_jmpnzero). & \\ \hline
50-51 & ин\-вер\-ти\-ро\-ван    & Проверить один разряд, и перейти, если не ноль (testbit\_jmpnzero). & \\ \hline
52-53 & ин\-вер\-ти\-ро\-ван    & Проверить один разряд векторного регистра, и перейти, если не ноль  (testbit\_jmpnzero). &  \\ \hline
54-57 & & Зарезервировано для последующего использования. & \\ \hline
58-59 & 0 переход \newline 1 вызов & Косвенно, с адресом указателя в регистре, указанном в RS, и смещением указателя в IM1 или IM2 (jump/call). & Форматы 1.4 и 2.7.0. \\ \hline
58-59 & 0 переход \newline 1 вызов & Безусловный косвенный переход/вызов с 16--разрядным или 32--разрядным смещением, либо с 64--разрядным абсолютным адресом  (jump/call). & Форматы 1.5 C, 2.7.2, и 3.0.1. \\ \hline
60-61 & 0 переход \newline 1 вызов & Использовать таблицу адресов относительно регистра, указанного в RD. RT = базовый адрес таблицы, RS = индекс*OS  (jump/call). & Формат 1.4, шаблон A. \\ \hline
60-61 & 0 переход \newline 1 вызов & Безусловный переход или вызов по адресу в регистре, указанном в RS  (jump/call).  & Формат 1.5. \\ \hline
62 & 0 & Возврат из функции (return). & Формат 1.4.  \\ \hline
62 & 0 & Возврат из системной функции (sys\_return). & Формат 1.5.  \\ \hline
63 & 1 & Системный вызов. ID --- в регистре, указанном в RT, блок разделяемой памяти --- в RD, длина --- в RS. Маски нет (sys\_call). & Формат 1.4, шаблон A. \\ \hline
63 & 1 & Системный вызов. ID --- в константах, блок разделяемой памяти --- в RD, длина --- в RS. Маски нет (sys\_call). & Форматы 2.7.1, 2.7.4 и 3.0.1. \\ \hline
63 & 1 & Безусловное прерывание (trap). Номер прерывания --- в IM1 (trap). & Формат 1.5. \\ \hline
63 & 1 & Заполнитель для неиспользуемой памяти кода. Все поля равны 1 (filler). & Формат 1.5. \\ \hline
63 & 1 & Прерывание, если (беззнаково) RD \textgreater{} IM3. IM2 = 38. Номер прерывания фиксирован (cmp\_unsign\_trapabove). & Формат 2.7.3. \\ \hline
\end{longtable}

Знаковые целочисленные сравнения корректируются при переполнении, а знаковые сложения и вычитания --- нет. Например, если A --- большое положительное целое число, а B --- большое отрицательное целое число, то  sub\_sign\_jmpneg выполнит переход, если вычисление  A-B из--за переполнения даст отрицательный результат, а cmp\_sign\_jmpbelow не выполнит переход, ибо A --- больше B.

Комбинирование команд АЛУ и команд условного перехода можно закодировать в форматах 1.4, 1.5 C, 2.7.0, 2.7.1, 2.7.2, 2.7.3, и 3.0.1, за исключением вычитания, которое не может быть закодировано в формате 1.5 C. Вычитание с непосредственно заданной константой в качестве операнда можно заменить сложением с отрицательной константой. Место в коде, которое в формате 1.5 C использовалось бы вычитанием, вместо этого используется для кодирования команды прямого перехода и вызова с 24--разрядным смещением, используя формат 1.5 D, в котором младшие три разряда поля OP1 используются как часть 24--разрядного смещения.

Операции сложения и вычитания для вещественных операндов обычно не поддерживаются, поскольку б\'{о}льшие задержки этих вещественных операций усложнят дизайн конвейера. Вещественные сравнения поддерживаются, поскольку можно выполнить операцию вещественного сравнения за один такт, используя беззнаковое целочисленное сравнение совместно экспоненты и мантиссы, со специальной обработкой знакового разряда и значений NAN.

Команда проверки разряда (testbit\_jmpnzero) проверяет в первом операнде разряд с номером n, где n --- значение во втором операнде (RS или IM2). Она полезна для проверки битовых полей, знаковых разрядов, и результата команд сравнения. Независимо от типа операнда, второй операнд интерпретируется как беззнаковое целое число.

Команды сдвига влево сдвигают первый операнд на количество разрядов, указанное вторым операндом, влево, если тот положителен, и вправо, с расширением нулём, если второй операнд отрицателен. Переносом является последний выдвинутый разряд. Независимо от типа операнда, операнды трактуются как целые, но в случае, если указан вещественный тип операнда (M = 1), --- используются векторные регистры.

Безусловные и косвенные переходы и вызовы используют указанные выше форматы, в которых неиспользуемые разряды должны быть нулями. Разряд №0 поля OPJ равен нулю для команд перехода, и единице для команд вызова.

\label{jumpTableInstruction}
Команды табличного косвенного перехода/вызова предназначены для облегчения реализации многопутёвых ветвлений (операторы switch/case), таблиц функций в интерпретаторах кода, и таблиц виртуальных функций в объектно--ориентированных языках с полиморфизмом. Таблица адресов перехода или вызова хранится в виде знакового смещения относительно произвольной точки отсчёта, которая может быть адресом таблицы, базовым адресом кода, или любой другой точкой отсчёта. Тип операнда указывает размер элемента таблицы. Обязательно должны поддерживаться 16--разрядные и 32--разрядные смещения, прочие размеры --- необязательно. Использование относительных адресов делает таблицу компактнее, нежели при использовании 64--разрядных абсолютных адресов. Команда работает следующим образом: вычисляется адрес элемента таблицы, как сумма базового указателя (RT) и индекса (RS), умноженного на размер операнда; по данному адресу читается знаковое значение, которое умножается на 4; то, что получилось, расширяется знаком до 64 разрядов и складывается с указанной точкой отсчёта (RD); по вычисленному адресу совершается переход или вызов. Индекс массива (RS) умножается на размер операнда, в то время как элементы таблицы умножаются на размер слова команды (4). Поддержка маски --- необязательна.

Таблицу, используемую командами табличного перехода/вызова, можно разместить в секции константных данных (CONST), что делает возможным использование в качестве точки отсчёта базового адреса таблицы, и улучшает безопасность, давая доступ к таблице только для чтения.

Когда используются соглашения вызова, сформулированные на с.~\pageref{functionCallingConventions}, командам возврата смещение в стеке не нужно.

\label{systemCallInstruction}
Системные вызовы для идентификации системных функций используют числа, ID, а не адреса. Этот ID является комбинацией ID модуля, идентифицирующего конкретный модуль системы или драйвер устройства, и ID функции, идентифицирующего конкретную функцию в этом модуле. Как ID модуля, так и ID функции --- либо оба --- 16--разрядные, либо оба --- 32--разрядные, так что суммарный размер ID системного вызова может достигать 64 разрядов. Командаe sys\_call имеет следующие варианты:

\begin{longtable}{|l|l|p{44mm}|p{44mm}|}
\caption{Варианты команды системного вызова}
\label{table:syscallInstructions}
\endfirsthead
\endhead
\hline
Формат  & Тип операнда & ID функции               & ID модуля                  \\ \hline
1.4     & 32 разряда   & разряды 0-15 регистра RT & разряды 16-31 регистра RT  \\ \hline
1.4     & 64 разряда   & разряды 0-31 регистра RT & разряды 32-63 регистра RT  \\ \hline
2.7.1   & 32 разряда   & разряды 0-15 поля IM2    & разряды 16-31 поля IM2     \\ \hline
2.7.4   & 64 разряда   & разряды 0-15 поля IM21   & разряды 0-31 поля IM3      \\ \hline
3.0.1   & 64 разряда   & разряды 0-31 поля IM2    & разряды 0-31 поля IM3      \\ \hline
\end{longtable}

Команде sys\_call можно указать блок памяти, разделяемый с системной функцией. Адрес блока памяти задаётся в регистре, указываемом полем RD, а длина --- в регистре, указываемом полем RS. Этот блок памяти, к которому у вызывающего должны быть права доступа, разделяется с системной функцией. Системные функции получат те же права на этот блок, что и вызывающий поток, т.е. те же права на чтение и/или запись. Это полезно для быстрой передачи данных между вызывающим и системной функцией. У вызывающей и вызываемой функций нет никакого совместного доступа ни к какой другой области памяти. Если и поле RD, и поле RS равны нулю (т.е. указан регистр r0), то никакой блок памяти не разделяется. Команда sys\_call формата 2.7.4 не может иметь никакого блока разделяемой памяти.

Параметры системных функций передаются в регистрах, следуя тем же соглашениям вызова, что обычные функции. Регистры, используемые для передачи параметров, обычно отличаются от регистров, указанных в полях RD, RS и RT. Параметры функций, не помещающиеся в регистры, должны находиться в блоке разделяемой памяти.

Ловушки работают подобно прерываниям. Безусловные ловушки имеют 8--разрядный номер прерывания в IM1, являющийся индексом в таблице векторов прерываний, первоначально находящейся по равному нулю абсолютному адресу. Команда безусловной ловушки для дополнительной информации может использовать IM2. Условная ловушка предназначена для проверки границ массивов. Номер прерывания фиксирован (решение о значении ещё не принято). Условная ловушка может (необязательно) поддерживать в IM2 иные коды условия, использующие те же самые коды, что и OPJ в табл.~\ref{table:controlTransferInstructions}.

Команда ловушки с единицами во всех разрядах всех полей (код операции ---  0x6FFFFFFF) может использоваться в качестве заполнителя неиспользуемых частей памяти кода.

\section{Назначение кодов операций}
Коды операций и форматы могут быть назначены новым командам в соответствии со следующими правилами.

\begin{itemize}
\item Многоформатные команды. Часто используемые команды, которым нужно поддерживать много различных типов операндов, режимов адресации, и форматов, используют большую часть (или все) из следующих форматов: 0.0-0.9, 2.0-2.5, 2.8-2.9, и (необязательно) 3.1 и 3.3 (если поддерживаются команды тройного размера). Во всех этих форматах используется одно и то же значение поля OP1. Поле OP2 обязано быть равно 0. Команды с немногими операндами--источниками идут первыми.

\item Малые команды. В малой версии доступны лишь некоторые из наиболее употребимых команд, поскольку место есть только для 32 малых команд. Как показано в табл.~\ref{table:tinyInstructionsGP} на с.~\pageref{table:tinyInstructionsGP}, команды упорядочиваются по количеству и типам операндов.

\item Команды передачи управления, т.е. переходы, ветвления, вызовы, и возвраты, могут быть закодированы как короткие команды с IL = 1, Mode = 4 или 5, и
OP1 от 0 до 63, либо как команды двойного размера, с IL = 2, Mode = 7, OP1 от 0 до 15, и (необязательно) как команды тройного размера, с IL = 3, Mode = 0, 
OP1 от 0 до 15. См. с.~\pageref{table:jumpInstructionFormats}.

\item Короткие одноформатные команды с регистрами общего назначения. Используйте форматы 1.0, 1.1, и 1.8, с любым значением поля OP1.

\item Короткие одноформатные команды с векторными регистрами. Используйте форматы 1.2 и 1.3, с любым значением поля OP1.

\item Одноформатные команды двойного размера с регистрами общего назначения могут использовать форматы 2.8 и 2.9, с любым значением полей OP1 и OP2 $\geq$ 8 (для одного и того же OP1 предоставляйте сходные команды), и формат 2.6 с любым значением поля OP1.

\item Одноформатные команды двойного размера с векторными регистрами могут использовать форматы 2.4 и 2.5, с любым значением полей OP1 и OP2 $\geq$ 8 (для одного и того же OP1 предоставляйте сходные команды), и формат 2.7 с OP1 в диапазоне от 16 до 63.

\item Одноформатные команды тройного размера с регистрами общего назначения могут использовать формат 3.0 с OP1 в диапазоне от 16 до 63.

\item Одноформатные команды тройного размера с векторными регистрами могут использовать формат 3.2 с любым значением поля OP1.

\item Последующие команды, с длиной, большей трёх 32--разрядных слов, кодируются с IL = 3, Mode = от 4 до 7.

\item Новые опции или другие модификации существующих команд могут использовать разряды OP3 или разряды регистра маски.

\item Новые режимы адресации могут быть реализованы как одноформатные команды чтения и записи. Новые режимы адресации или иные модификации, которые применимы ко всем мультиформатным командам, могут использоваь для разрядов опций поле OP3. Если разрядов поля OP3 недостаточно, то возможно, в качестве последнего средства, использование значений поля OP2 из диапазона от 1 до 7.
\end{itemize}

Все неиспользуемые поля должны быть равны нулю. Для команд с наименьшим количеством входных операндов следует предпочесть наименьшие значения кодов в OP1.

Операнды назначаются следующим образом. Операнд--приёмник представляет собой регистр, указанный в поле RD. Операнды--источники используют регистровые поля RS, RT и RU, если только эти поля не задействованы для других целей (например, базового указателя, индекса, длины вектора). Если имеется находящийся в памяти операнд, либо непосредственно заданный операнд, то он должен быть последним из операндов--источников. Если выбранный формат имеет меньше операндов--источников, нежели необходимо для команды, то поле RD используется и как операнд--приёмник, и как первый из операндов--источников. Если всё ещё недостаточно операндов, то формат для конкретной команды использоваться не может. Если в формате имеется больше операндов, чем необходимо, то любой находящийся в памяти операнд или непосредственно заданный операнд будет последним операндом--источником, имея приоритет перед любым регистровым операндом. Неиспользуемые поля операндов должны быть равны нулю.
\end{document}