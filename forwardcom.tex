\documentclass[10pt]{report}

% \usepackage[utf8]{inputenc}
\usepackage[utf8x]{inputenc}
\usepackage{hyperref}
%\usepackage{graphicx}
%\usepackage{float}
\usepackage[russian]{babel}
%\usepackage{listings}
%\lstset{inputencoding=utf8,extendedchars=false,keepspaces=true} %,language=C++

\usepackage{longtable}
\usepackage{multirow}
\usepackage{multicol}
\usepackage{ccaption}
\usepackage{verbatim}
\usepackage{amsfonts}
\usepackage{amssymb}
\usepackage{amsmath}
\usepackage{indentfirst}
\usepackage{subfiles}
\usepackage{syntax}
\usepackage{bytefield}



\usepackage{algpseudocode}
\usepackage{graphicx}

\usepackage[dvipsnames,usenames]{color}
\usepackage{hyperref}
\hypersetup{unicode,breaklinks=true}
%\usepackage{tikz}
%\usetikzlibrary{arrows,decorations.pathmorphing,backgrounds,positioning,fit,petri,calc,shapes.misc,shapes.geometric}
%\usepackage{minted}

\textheight 24.0cm
\textwidth 16.8cm

\voffset -2cm
\hoffset .0cm
\oddsidemargin 0.5mm
\evensidemargin 0.5mm
\topmargin -0.4mm
\righthyphenmin=2
\hfuzz=12.7pt
\makeatletter
%\renewcommand\section{\@startsection {section}{1}{\z@}%
%                                   {3.5ex \@plus 1ex \@minus .2ex}%
%                                   {2.3ex \@plus.2ex}%
%                                   {\normalfont\Large\bfseries}}
%\renewcommand{\thesection}{\arabic{section}}
\@addtoreset{equation}{section}
\@addtoreset{figure}{section}
\@addtoreset{table}{section}

\newlength{\chaprule}
\newlength{\ATchapskip}
\setlength{\chaprule}{0.4\p@} \setlength{\ATchapskip}{10\p@} \advance \ATchapskip by -1\chaprule
\renewcommand{\@makechapterhead}[1]{%
	\ifdim\chaprule>6\p@ \setlength{\chaprule}{6\p@}\fi
	\vspace*{\ATchapskip}%
	%\noindent\rule{\textwidth}{\chaprule}\par%
	%\vskip 10\p@
	{\parindent \z@ \normalfont \ifnum \c@secnumdepth >\m@ne {\huge\bfseries \@chapapp\space \thechapter.}
		%\par\nobreak
		%\vskip 8\p@
		\fi \interlinepenalty\@M \huge\bfseries #1\par\nobreak \vskip 10\p@
		%\noindent\rule{\textwidth}{\chaprule}\par%
		\vskip\ATchapskip }}

%\@addtoreset{lstlisting}{section}
\makeatother


\newcounter{rem}[section]
\renewcommand{\therem}{\thesection.\arabic{rem}}
\newenvironment{Remark}{\par\refstepcounter{rem} \bf Замечание \therem. \sl}{\rm\par}
\renewcommand{\theequation}{\thesection.\arabic{equation}}
\renewcommand{\thefigure}{\thesection.\arabic{figure}}
\renewcommand{\thetable}{\thesection.\arabic{table}}

\renewcommand{\theenumi}{\arabic{enumi}}
\renewcommand{\labelenumi}{\theenumi)}
\newcommand{\udc}[1]{УДК #1}

\newcounter{lem}[section]
\renewcommand{\thelem}{\thesection.\arabic{lem}}
\newenvironment{Lemma}{\par\refstepcounter{lem}\bf Лемма \thelem. \sl}{\rm\par}

\newcounter{cor}[section]
\renewcommand{\thecor}{\thesection.\arabic{cor}}
\newenvironment{Corrolary}{\par\refstepcounter{cor}\bf Следствие \thecor. \sl}{\rm\par}
\newcounter{theor}[section]
\renewcommand{\thetheor}{\thesection.\arabic{theor}}
\newenvironment{Theorem}{\par\refstepcounter{theor}\bf Теорема \thetheor. \sl}{\rm\par}
%\let \kappa=\ae
\newcommand{\diag}{\mathop{\rm diag}}
\newcommand{\epi}{\mathop{\rm epi}}
\newenvironment{Proof}{\par\noindent\bf Доказательство.\rm}{ \par}

\newcounter{exam}[section]
\renewcommand{\theexam}{\thesection.\arabic{exam}}
\newenvironment{Example}{\par\refstepcounter{exam}\bf Пример \theexam. \sl}{\rm\par}

\newcounter{prob}[section]
\renewcommand{\theprob}{\thesection.\arabic{prob}}
\newenvironment{Problem}{\par\refstepcounter{prob}\bf Задача \theprob. \sl}{\rm\par}

\newcounter{sol}[section]
\renewcommand{\thesol}{\thesection.\arabic{sol}}
\newenvironment{Solution}{\par\refstepcounter{sol}\bf Решение. \rm}{\rm\par}

\newcounter{defin}[section]
\renewcommand{\thedefin}{\thesection.\arabic{defin}}
\newenvironment{Definition}{\par\refstepcounter{defin}\bf Определение \thedefin.\sl}{\rm\par}

\newcounter{answ}[section]
\renewcommand{\theansw}{\thesection.\arabic{answ}}
\newenvironment{Answer}{\par\refstepcounter{answ}\theansw. \rm}{\rm\par}

\newcounter{exerc}[section]
\renewcommand{\theexerc}{\thesection.\arabic{exerc}}
\newenvironment{Exercise}{\par\refstepcounter{exerc}\bf Упражнение \theexerc.\sl}{\rm\par}

\newcounter{algor}[section]
\renewcommand{\thealgor}{\thesection.\arabic{algor}}
\newenvironment{Algorithm}[3]{\par\refstepcounter{algor}{\bf Алгоритм \thealgor. }{\sl #1}\par{\bf Вход: }{\sl #2}\par{\bf Выход: }\sl #3\par{\bf Метод.}\par}{\rm\par}

\newcommand{\ljoq}{<<}
\newcommand{\rjoq}{>>}
\captiondelim{. }



%\setcounter{section}{1}


\makeatletter
\def\@seccntformat#1{\csname the#1\endcsname.\quad}
\makeatother

\algrenewcommand\algorithmicwhile{\textbf{пока}}
\algrenewcommand\algorithmicdo{\textbf{ }}
\algrenewcommand\algorithmicend{\textbf{конец}}
\algrenewcommand\algorithmicif{\textbf{если}}
\algrenewcommand\algorithmicthen{\textbf{то}}
\algrenewtext{EndIf}{\textbf{всё}}
\algrenewcommand\algorithmicfor{\textbf{для}}
\algrenewtext{ForAll}{\textbf{для всех }}
\algrenewcommand\algorithmicelse{\textbf{иначе}}
\algrenewtext{ElsIf}{\textbf{инес }}
\algrenewtext{Procedure}{\textbf{алг }}
\algrenewtext{EndProcedure}{\textbf{кон }}

% modify style
\renewcommand{\familydefault}{\sfdefault}
% \setlength{\parindent}{1cm} % doesn't work.

\begin{document}
\title{ForwardCom: открытый стандарт набора команд для высокопроизводительных микропроцессоров}
\author{Агнер Фог}
\date{\today}
\maketitle
%\raggedright
%\RaggedRight

\tableofcontents
\setcounter{secnumdepth}{1}

% Введение
% - Основные моменты
% - Основы
% - Цели дизайна
% - Сравнение с другими открытыми наборами команд
% - Литература и ссылки
\subfile{fwc-introduction.tex}

% Основы архитектуры
% - Полностью ортогональный набор команд
% - Размер команды
% - Набор регистров
% - Поддержка векторов
% - Циклы по векторам
% - Максимальная длина вектора
% - Маски команд
% - Режимы адресации
\subfile{fwc-basic-architecture.tex}
%
% Форматы команд
% - Форматы и шаблоны
% - Кодирование операндов
% - Кодирование масок
% - Формат команд перехода, вызова, и ветвления
% - Назначение кодов операций
\subfile{fwc-instruction-formats.tex}
%
% Списки команд
% - Список многоформатных команд
% - Список малых команд
% - Список одноформатных команд
\subfile{fwc-instruction-lists.tex}
%
% - Описание команд
% - Распространённые операции, для которых нет специальных команд
% - Неиспользуемые команды
\subfile{fwc-description-of-instructions.tex}
%
% Другие детали реализации
% - Порядок байтов
% - Реализация стека вызовов
% - Вещественные ошибки и исключения
% - Обнаружение целочисленного переполнения
% - Многопоточность
% - Свойства безопасности
\subfile{fwc-other-implementation-details.tex}
%
%% Programmable application-specific instructions
%\subfile{fwc_programmable_instructions.tex}
%
%% Microarchitecture and pipeline design
%\subfile{fwc_microarchitecture_and_pipeline.tex}
%
%% Memory model
%% - Thread memory protection
%% - Memory management
%\subfile{fwc_memory_model.tex}
%
%% System programming
%% - Memory map
%% - Call stack
%% - System calls and system functions
%% - Inter-process calls
%% - Error message handling
%\subfile{fwc_system_programming.tex}
%
%% Standardization of ABI and software ecosystem
%% - Compiler support
%% - Binary data representation
%% - Further conventions for object-oriented languages
%% - Function calling convention
%% - Register usage convention
%% - Name mangling for function overloading
%\subfile{fwc_abi_standard.tex}
%
%% - Binary format for object files and executable files
%% - Function libraries and link methods
%% - Library function dispatch system
%% - Predicting the stack size
%% - Exception handling, stack unrolling and debug information
%\subfile{fwc_object_file_format.tex}
%
%% - Assembly language syntax
%\subfile{fwc_assembly_language_syntax.tex}
%
%% Conclusion
%\subfile{fwc_conclusion.tex}
%
%% Revision history
%\subfile{fwc_revision_history.tex}
%
%% Copyright notice
%\subfile{fwc_copyright_notice.tex}
\end{document}
